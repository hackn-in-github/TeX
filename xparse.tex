%#!ptex2pdf -u -l -od "-p a4 -r 1200" xparse
%#!uplatex xparse
%#LPR dvipdfmx -p a4 -r 1200 xparse
%#MAKEINDEX mendex -g -s mystyle.ist xparse
% ~/texmf/makeindex/base/mystyle.ist
%#!uplatex -shell-escape xparse
%#!platex -shell-escape xparse
%#!platex xparse
%%% sudo tlmgr update --self --all
% ptex2pdf option xparse
%    -v version
%    -h help
%    --help (help の完全版で、TeXworks での設定の説明も含む)
%    -e use eptex class of programs(euptex, eptex を使用する場合に指定)
%    -u use uptex class of programs(uplatex, euptex, uptex を使用する場合に指定)
%    -l use latex based formats(uplatex, platex を使用する場合に指定)
%    -s stop at dvi(PDF を作成しない)
%    -i retain intermediate files(dvi を削除しない)
%    -ot '<opts>' extra options for TeX(TeX コンパイラのオプション指定)
%    -od '<opts>' extra options for dvipdfmx(dvipdfmx のオプション指定)
\documentclass[a4j,uplatex,dvipdfmx]{jsbook}
%\usepackage{etex}%32768個のレジスタを使えるようにする
% TeX Live 2015 から不要になる
\usepackage{hyperref,pxjahyper,xparse,plext,amsmath,graphicx,tikz,array,hhline,cases}
\hypersetup{%
	pdfstartview={FitH -32768},%横幅に合わせて表示
	bookmarks=true,%しおりを付ける
	bookmarksnumbered=true,%章や節の番号を振る
	bookmarkstype=toc,%目次情報ファイル(\jobname.toc)を参照
	linkbordercolor={0 1 1},% default 1 0 0
	colorlinks=false,% ハイパーリンクを文字に色を付けるのではなく色枠に
	citebordercolor={0 1 0},% cite の枠の色 lime [default]
	urlbordercolor={0 0 1},% url の枠の色 blue [default]
%	bookmarksopen=true,%しおりの展開(Not suported by dvipdfmx)
%	bookmarksopenlevel=3,%目次の深さを指定(Not suported by dvipdfmx)
%	breaklinks=true,%改行を含むリンクを許す(Not suported by dvipdfmx)
	pdftitle={xparse},%
%	pdfsubtitle={文書のプロパティのサブタイトル},%
	pdfauthor={飯島 徹},%
%	pdfkeywords={文書のプロパティのキーワード},%
	}
\usetikzlibrary{arrows,shapes.symbols,decorations.pathreplacing,%
	patterns,topaths,through,calc,intersections,%
	shapes.geometric,shapes.misc,shapes.multipart,chains,angles,quotes,%
	arrows.meta}
\usepackage[framemethod=tikz]{mdframed}
\let\TikZforeach=\foreach
\usepackage{emath,emathMw,emathEy,itembbox,itemtopmath,EMproof}
\let\EMforeach=\foreach
\let\foreach=\TikZforeach
\let\therefore=\relax
\let\because=\relax
\usepackage{mathabx}
%emathとmathabxを読み込む時のエラー対策
%\usepackage{newtxtext,newtxmath}
\usepackage{tikzsetup}
\pagestyle{myfoot}
\newdimen\zw
\newdimen\zh
\zw=1zw\relax
\zh=1zh\relax
\newdimen\tempdima
\makeatletter
\makeatother
%%\usepackage{multicol}%任意の位置で段組
%%\begin{multicols}{2}
%%\end{multicols}
%%で均等割の2段組
%%\begin{multicols*}{2}
%%\end{multicols*}
%%で詰め込みの2段組
%% ラテン筆記体
%\usepackage{mathrsfs,calligra}%
% \mathscr{I},\textxalligra{I}
%\usepackage[dvipdfmx]{graphicx,color}
%\問題hidden
%\解説hidden
%\CenterLine=1pt% 囲み罫に使う線の幅(\Center)
%\refCenterLine=.4pt\relax% 相互参照の囲み罫に使う線の幅(\refMark)
%\CenterSpc=3.5pt% 囲み罫と中の文字との幅(\Center)
%\CenterBoxWidth=3zw% 囲み罫の幅(\Center)
%f'(x)>0 and f''(x)>0 \nevarrow
%f'(x)>0 and f''(x)<0 \necarrow
%f'(x)<0 and f''(x)>0 \sevarrow
%f'(x)<0 and f''(x)<0 \secarrow
%\futurelet#1#2#3 -> #1に#3を代入して#2を実行するが,#3はそのままにしておく
%                    その際,#3は`{'であっても構わず実行する
%                    すなわち,\futurelet\token\hoge{\fuga}は
%                    \let\token={を行なってから\hogeを実行する
%                    \let\token=\fugaにはならないので注意
% ex: \def\判定{\ifthenelse{\equal{\hoge}{\fuga}}{\fuga と一致}{\fugaと不一致}}%\ifx\hoge\fuga \fuga と一致\else\fuga と不一致\fi}%
%     \let\fuga=/%
%     \futurelet\hoge\判定/
%     \futurelet\hoge\判定*
%     \let\fuga={%
%     \def\awawa{``これが判定されるわけではない''}%
%     \futurelet\hoge\判定{\awawa}%\awawaの前の`{'が判定される
%\Vecの引数に\Markが入るときは\Vecを\overrightarrowに置き換える必要がある.
\everymath{\displaystyle}
%\listfiles
%\MarkAnswerfalse
%\NoHRSectiontrue
% 改行:C-q C-j
% 改行+復帰:C-q C-j C-q C-m
% タブ:C-q C-i
%\nointerlineskip
%\stackrel#1#2
% #2の上に#1を乗せる(#2は普通の大きさ,#1は小さくなる)
%\numexpr,\dimexpr,\glueexpr,\muexpr(e-tex拡張)
%\setexamindex{}
%\includeonly{IA/2007-Center-IA-4-tikz}
%\resettagform
\begin{document}
\begin{verbatim}
\NewDocumentCommand\hoge{m m}{%
 {\Huge #1}{\tiny #2}%
}%
\hoge{あいうえお}{かきくけこ}%
\end{verbatim}
\NewDocumentCommand\hoge{m m}{%
 {\Huge #1}{\tiny #2}%
}%

\hoge{あいうえお}{かきくけこ}%

\begin{verbatim}
\RenewDocumentCommand\hoge{o m}{%
 \IfValueT{#1}{{\Huge #1}}%
 {\tiny #2}%
}%

\hoge{かきくけこ}%

\hoge[あいうえお]{かきくけこ}%
\end{verbatim}
\RenewDocumentCommand\hoge{o m}{%
 \IfValueT{#1}{{\Huge #1}}%
 {\tiny #2}%
}%

\hoge{かきくけこ}%

\hoge[あいうえお]{かきくけこ}%

\begin{verbatim}
\RenewDocumentCommand\hoge{O{あいうえお} m}{%
 {\Huge #1}{\tiny #2}%
}%

\hoge{かきくけこ}%

\hoge[さしすせそ]{かきくけこ}%
\end{verbatim}
\RenewDocumentCommand\hoge{O{あいうえお} m}{%
 {\Huge #1}{\tiny #2}%
}%
\hoge{かきくけこ}%

\hoge[さしすせそ]{かきくけこ}%

\begin{verbatim}
\RenewDocumentCommand\hoge{r() m}{%
 {\Huge #1}{\tiny #2}%
}%

\hoge(あいうえお){かきくけこ}%
\end{verbatim}
\RenewDocumentCommand\hoge{r() m}{%
 {\Huge #1}{\tiny #2}%
}%

\hoge(あいうえお){かきくけこ}%

\begin{verbatim}
\RenewDocumentCommand\hoge{D(){あいうえお} m}{%
 {\Huge #1}{\tiny #2}%
}%

\hoge{かきくけこ}%

\hoge(さしすせそ){かきくけこ}%
\end{verbatim}
\RenewDocumentCommand\hoge{D(){あいうえお} m}{%
 {\Huge #1}{\tiny #2}%
}%

\hoge{かきくけこ}%

\hoge(さしすせそ){かきくけこ}%

\begin{verbatim}
\RenewDocumentCommand\hoge{D<>{あいうえお} m}{%
 {\Huge #1}{\tiny #2}%
}%

\hoge{かきくけこ}%

\hoge<さしすせそ>{かきくけこ}%
\end{verbatim}
\RenewDocumentCommand\hoge{D<>{あいうえお} m}{%
 {\Huge #1}{\tiny #2}%
}%

\hoge{かきくけこ}%

\hoge<さしすせそ>{かきくけこ}%

\unitlength=1mm\relax
\begin{picture}(100,80)(-15,-15)
\put(-15,-15){\framebox(100,80){}}
\put(0,0){\vector(1,0){80}}
\put(0,0){\vector(0,1){60}}
\put(24,-10){\footnotesize 海賊の数}
\put(-11,8){\footnotesize\pbox<t>{地球平均気温}}
\end{picture}
\end{document}